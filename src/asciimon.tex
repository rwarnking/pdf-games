\documentclass[12pt, a4paper, footexclude, headexclude]{scrartcl}

%%%%%%%%%%%%%%%%%%%%%%%%%%%%%%%%%%%%%%%%%%%%%%%%%%%%%%%%%%%%%%%%%%%%%%%%%%%%%%%%%%%%%%%%%%%%%%%%%%%
% Packages
%%%%%%%%%%%%%%%%%%%%%%%%%%%%%%%%%%%%%%%%%%%%%%%%%%%%%%%%%%%%%%%%%%%%%%%%%%%%%%%%%%%%%%%%%%%%%%%%%%%
\usepackage[margin=0.5in]{geometry}

% \usepackage[utf8x]{inputenc}
% \usepackage[T1]{fontenc}
% \usepackage[ngerman]{babel}
% \usepackage{xcolor}
\usepackage[pdftex]{hyperref}
\usepackage{multicol}

\usepackage{xinttools}
\usepackage[nomessages]{fp}
\usepackage{tabularx}
\usepackage{tcolorbox}

\setlength{\parindent}{0pt}

\usepackage[pdftex]{insdljs}

%%%%%%%%%%%%%%%%%%%%%%%%%%%%%%%%%%%%%%%%%%%%%%%%%%%%%%%%%%%%%%%%%%%%%%%%%%%%%%%%%%%%%%%%%%%%%%%%%%%
% Parameter
%%%%%%%%%%%%%%%%%%%%%%%%%%%%%%%%%%%%%%%%%%%%%%%%%%%%%%%%%%%%%%%%%%%%%%%%%%%%%%%%%%%%%%%%%%%%%%%%%%%
% Obacht: It is discuraged to use a field with size bigger than 15x15 since the pdf might get
% to slow to play
\def\fieldWidth{10}
\def\fieldHeight{10}

% Max playercount is 5
\def\playerCount{5}

%%%%%%%%%%%%%%%%%%%%%%%%%%%%%%%%%%%%%%%%%%%%%%%%%%%%%%%%%%%%%%%%%%%%%%%%%%%%%%%%%%%%%%%%%%%%%%%%%%%
% Javascript
%%%%%%%%%%%%%%%%%%%%%%%%%%%%%%%%%%%%%%%%%%%%%%%%%%%%%%%%%%%%%%%%%%%%%%%%%%%%%%%%%%%%%%%%%%%%%%%%%%%
\begin{insDLJS}[]{JSDateiname}{asciimon}

const GAME_DESC = "Text.\
Text.\
Text.\
Text.\
Text.";

% Outside variables
const FIELD_WIDTH = AFMakeNumber(\fieldWidth);
const FIELD_HEIGHT = AFMakeNumber(\fieldHeight);

% Constants

% Globals
% The game field
%var complField = new Array();

% Color data
const COLORS = new Array(
    %[ "RGB", 0.843, 0.078, 0.058 ],
    %[ "RGB", 0.203, 0.603, 0.054 ],
    %[ "RGB", 0.078, 0.305, 0.560 ]
);

const ANIMAL_COUNT = 2;
const A_NAME = 0;
const A_HP = 1;
const A_TYPE = 2;
const A_ATK = 3; % attacks

const T_WATER = 0;
const T_AIR = 1;

const ANIMAL_LIST = new Array(
    ["SHEEP", 68, T_WATER, [0, 1, 2, 3]],
    ["BIRD", 68, T_WATER, [0, 1, 2, 3]]
);

const ATTACK_LIST = new Array(
    "JUMP", "CRY", "SPEAK", "LOOK"
);

%%%%%%%%%%
% Helper %
%%%%%%%%%%
/**
 * Returns a random number between 0 and the param.
 * @param {Number} max The upperbound
 */
function getRandomInt(max) {
    return Math.floor(Math.random() * max);
}

%%%%%%%%%%%%%%%%%%%%%%%%%%%%%%
% Initialisation and Restart %
%%%%%%%%%%%%%%%%%%%%%%%%%%%%%%
/**
 * Initialise the default game state.
 */
function initialise() {
    % Init the GUI
    initialiseButtons();

    getEnemy();

    drawAll();
}

/**
 * Initialise all buttons from the GUI.
 * For example add callbacks and set the correct colors.
 */
function initialiseButtons() {
    % Assign each button the coresponding function and disable the highlights
    this.getField("restart").setAction("MouseUp", "restart();");
    this.getField("restart").highlight="none";

    % Info field setup
    this.getField("info").fillColor = color.transparent;
    this.getField("info").borderColor = color.transparent;

    % Tooltip setup
    this.getField("tooltipbtn").setAction("MouseDown", "tooltip(true);");
    this.getField("tooltipbtn").setAction("MouseUp", "tooltip(false);");
    this.getField("tooltipbtn").highlight="none";

    printInfoText("Init");
}

/**
 * Restart the game and clear everything
 */
function restart() {
    printInfoText("Restart");
}

/**
 * Enable or disable the tooltip
 *
 * @param {Boolean} on Enable or disable tooltip
 */
function tooltip(on) {
    if (on) {
        % Move the tooltip field to the center of the screen
        % and adjust the size so the text can be displayed
        var field = this.getField("tooltiptxt");
        field.delay = true;
        field.hidden = false;
        field.value = GAME_DESC;

        field.multiline = true;
        field.textSize = 16;
        field.fillColor = color.gray;
        % height 842, width 596
        % upper-left x, upper-left y, lower-right x and lower-right y
        field.rect = [128, 601, 468, 401];

        field.delay = false;
    } else {
        % Remove the field such that it is not visible
        var field = this.getField("tooltiptxt");
        field.delay = true;
        field.hidden = true;
        field.value = "";

        field.multiline = false;
        field.fillColor = color.gray;
        field.rect = [0, 0, 0, 0];

        field.delay = false;
    }
}

%%%%%%%%%%%%%
% Gamelogic %
%%%%%%%%%%%%%

function getPlayer() {

}

function getEnemy() {
    var enemy = getRandomInt(ANIMAL_LIST.length);

    this.getField("enemyname").value = ANIMAL_LIST[enemy][A_NAME];
    this.getField("enemystatus").value = ANIMAL_LIST[enemy][A_HP];
    this.getField("enemyhp").value = ANIMAL_LIST[enemy][A_TYPE];
}

%%%%%%%%%%%%%%%%%%%%%
% Draw and coloring %
%%%%%%%%%%%%%%%%%%%%%
/**
 * Draw all cells.
 */
function drawAll() {
    for (var m = 0; m < FIELD_HEIGHT; m++) {
        for (var n = 0; n < FIELD_WIDTH; n++) {
            draw(m, n);
        }
    }
}

/**
 * Since only one cell is updated with each input,
 * only this cell needs to be redrawn.
 *
 * @param {Number} m The m position.
 * @param {Number} n The n position.
 */
function draw(m, n) {
    var cell = this.getField("cell" + m + "-" + n);
    cell.delay = true;
    cell.value = "tmp";
    cell.delay = false;
}

/**
 * Print information about the winner and the gamestate.
 */
function printInfoText(text) {
    var field = this.getField("info");
    field.delay = true;
    field.value = text;
    field.delay = false;
}

\end{insDLJS}

\OpenAction{/S /JavaScript /JS (initialise();)}

%%%%%%%%%%%%%%%%%%%%%%%%%%%%%%%%%%%%%%%%%%%%%%%%%%%%%%%%%%%%%%%%%%%%%%%%%%%%%%%%%%%%%%%%%%%%%%%%%%%
% Dokument
%%%%%%%%%%%%%%%%%%%%%%%%%%%%%%%%%%%%%%%%%%%%%%%%%%%%%%%%%%%%%%%%%%%%%%%%%%%%%%%%%%%%%%%%%%%%%%%%%%%
\begin{document}

    \begin{Form}

        % Header
        \begin{multicols}{2}
            % Title
            \section*{ASCIImon}
        \columnbreak
            % Tooltip object
            \begin{flushright}%
                \PushButton[name=tooltipbtn, bordercolor=white]{%
                    \begin{tcolorbox}[width=20pt, height=20pt, left=3pt, top=0pt]
                        \centering?\strut{}
                    \end{tcolorbox}%
                }
            \end{flushright}%
        \end{multicols}

        % Calculate the max width and height of a block
        \FPeval{\resultW}{(\fieldWidth * 3)}
        \def\blocksizeW{\dimexpr ((\linewidth)-\resultW pt)/\fieldWidth \relax}

        \FPeval{\resultH}{(\fieldHeight * 3)}
        \def\blocksizeH{\dimexpr (250 pt -\resultH pt)/\fieldHeight \relax}

        % Define the blocksize via the smaller amount
        \ifnum\blocksizeW<\blocksizeH
            \def\blocksize{\blocksizeW}
        \else
            \def\blocksize{\blocksizeH}
        \fi

        \begin{multicols}{2}

            \begin{tcolorbox}%
                \TextField[name=enemyname, width=\linewidth, readonly=true]{}\newline

                \TextField[name=enemystatus, width=\linewidth, readonly=true]{}\newline

                \TextField[name=enemyhp, width=\linewidth, readonly=true]{}%
            \end{tcolorbox}%

            \vspace{2.5cm}

            \begin{center}
                \xintFor* #2 in {\xintSeq{0}{\fieldHeight-1}} \do {%
                    \xintFor* #1 in {\xintSeq{0}{\fieldWidth-1}} \do {%
                        \TextField[%
                            width=\blocksize, height=\blocksize, bordercolor=black, name=player#1-#2, readonly=true%
                        ]{}\hspace{3pt}%
                    }%
                    \\[2pt]%
                }%
            \end{center}

        \columnbreak%

            \begin{center}
                \xintFor* #2 in {\xintSeq{0}{\fieldHeight-1}} \do {%
                    \xintFor* #1 in {\xintSeq{0}{\fieldWidth-1}} \do {%
                        \TextField[%
                            width=\blocksize, height=\blocksize, bordercolor=black, name=enemy#1-#2, readonly=true%
                        ]{}\hspace{2pt}%
                    }%
                    \\[2pt]%
                }%
            \end{center}

            \vspace{2.5cm}

            \begin{tcolorbox}%
                \TextField[name=playername, width=\linewidth, readonly=true]{}\newline

                \TextField[name=playerstatus, width=\linewidth, readonly=true]{}\newline

                \TextField[name=playerhp, width=\linewidth, readonly=true]{}%
            \end{tcolorbox}%

        \end{multicols}

        %%%%%%%%%%%%%%%%%%%%%%%
        % Commandredifinition %
        % https://www.dickimaw-books.com/latex/admin/html/eforms.shtml
        %%%%%%%%%%%%%%%%%%%%%%%
        % \def\DefaultHeightofText{14pt}
        % \renewcommand*{\LayoutTextField}[2]{%
        %     \parbox[c][\DefaultHeightofText]{0.5\linewidth}{#1#2}%
        % }

        % \renewcommand*{\LayoutCheckField}[2]{#1 #2}
        % \renewcommand*{\DefaultWidthofCheckBox}{2ex}
        % \renewcommand*{\DefaultHeightofCheckBox}{2ex}
        % \renewcommand*{\LayoutCheckField}[2]{%
        %     \parbox[c][\DefaultHeightofCheckBox]{0.12\linewidth}{#1}\enspace%
        %     \parbox[c][\DefaultHeightofCheckBox]{\DefaultWidthofCheckBox}{#2}%
        % }

        % \renewcommand*{\DefaultWidthofChoiceMenu}{2.5ex}
        % \renewcommand*{\DefaultHeightofChoiceMenu}{2.04ex}

        %%%%%%%%%%%%%%%%
        % GUI elements %
        %%%%%%%%%%%%%%%%
        \begin{center}
            \begin{tcolorbox}
                \TextField[name=info, width=\linewidth, readonly=true]{}
            \end{tcolorbox}

            % Game related buttons
            \begin{tabularx}{\textwidth}{@{} *{2}{X} @{}}%
                \PushButton[name=fight, bordercolor=white, borderwidth=0]{
                    \begin{tcolorbox}
                        \centering
                        Fight\strut
                    \end{tcolorbox}
                } &
                \PushButton[name=change, bordercolor=white, borderwidth=0]{
                    \begin{tcolorbox}
                        \centering
                        Change\strut
                    \end{tcolorbox}
                }\\
                \PushButton[name=item, bordercolor=white, borderwidth=0]{
                    \begin{tcolorbox}
                        \centering
                        Item\strut
                    \end{tcolorbox}
                } &
                \PushButton[name=run, bordercolor=white, borderwidth=0]{
                    \begin{tcolorbox}
                        \centering
                        Run\strut
                    \end{tcolorbox}
                }\\
                \PushButton[name=back, bordercolor=white, borderwidth=0]{
                    \begin{tcolorbox}
                        \centering
                        Back\strut
                    \end{tcolorbox}
                } &
                \PushButton[name=restart, bordercolor=white, borderwidth=0]{
                    \begin{tcolorbox}
                        \centering
                        Restart game\strut
                    \end{tcolorbox}
                }
                \\
            \end{tabularx}
        \end{center}

        \TextField[bordercolor=white, name=tooltiptxt, readonly=true, hidden]{}

    \end{Form}
\end{document}
